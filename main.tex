\documentclass[a4paper, 12pt]{amsart}

%\usepackage{etoolbox}
%\makeatletter
%\let\ams@starttoc\@starttoc
%\makeatother
%\makeatletter
%\let\@starttoc\ams@starttoc
%\patchcmd{\@starttoc}{\makeatletter}{\makeatletter\parskip\z@}{}{}
%\makeatother

%\usepackage[parfill]{parskip}
\usepackage{vmargin}
\usepackage[colorlinks=true,linkcolor=blue,citecolor=blue,urlcolor=blue]{hyperref}
\usepackage{bookmark}
\usepackage{amsthm,thmtools,amssymb,amsmath,amscd,amsfonts}
\usepackage{mathrsfs}
\usepackage{stmaryrd}


\usepackage[bibstyle=authoryear,citestyle=authoryear,backend=bibtex]{biblatex}
\bibliography{refs}

\usepackage{fancyhdr}
\usepackage{esint}

\usepackage{enumerate}

\usepackage{pictexwd,dcpic}

\usepackage{graphicx}
\usepackage[utf8]{inputenc}

\declaretheorem[name=Theorem,numberwithin=section]{thm}
\declaretheorem[name=Remark,style=remark,sibling=thm]{rem}
\declaretheorem[name=Lemma,sibling=thm]{lemma}
\declaretheorem[name=Proposition,sibling=thm]{prop}
\declaretheorem[name=Definition,style=definition,sibling=thm]{defn}
\declaretheorem[name=Corollary,sibling=thm]{cor}
\declaretheorem[name=Assumption,style=remark,sibling=thm]{ass}
\declaretheorem[name=Example,style=remark,sibling=thm]{example}


\numberwithin{equation}{section}

\usepackage{cleveref}
\crefname{lemma}{Lemma}{Lemmata}
\crefname{prop}{Proposition}{Propositions}
\crefname{thm}{Theorem}{Theorems}
\crefname{cor}{Corollary}{Corollaries}
\crefname{defn}{Definition}{Definitions}
\crefname{example}{Example}{Examples}
\crefname{rem}{Remark}{Remarks}
\crefname{ass}{Assumption}{Assumptions}
\crefname{not}{Notation}{Notation}
\crefname{section}{Section}{Sections}

%Symbols
\renewcommand{\~}{\tilde}
\renewcommand{\-}{\bar}
\newcommand{\bs}{\backslash}
\newcommand{\cn}{\colon}
\newcommand{\sub}{\subset}

\newcommand{\N}{\mathbb{N}}
\newcommand{\R}{\mathbb{R}}
\newcommand{\Z}{\mathbb{Z}}
\renewcommand{\S}{\mathbb{S}}
\renewcommand{\H}{\mathbb{H}}
\newcommand{\C}{\mathbb{C}}
\newcommand{\K}{\mathbb{K}}
\newcommand{\Di}{\mathbb{D}}
\newcommand{\B}{\mathbb{B}}
\newcommand{\8}{\infty}

%Greek letters
\renewcommand{\a}{\alpha}
\renewcommand{\b}{\beta}
\newcommand{\g}{\gamma}
\renewcommand{\d}{\delta}
\newcommand{\e}{\epsilon}
\renewcommand{\k}{\kappa}
\renewcommand{\l}{\lambda}
\renewcommand{\o}{\omega}
\renewcommand{\t}{\theta}
\newcommand{\s}{\sigma}
\newcommand{\p}{\varphi}
\newcommand{\z}{\zeta}
\newcommand{\vt}{\vartheta}
\renewcommand{\O}{\Omega}
\newcommand{\D}{\Delta}
\newcommand{\G}{\Gamma}
\newcommand{\T}{\Theta}
\renewcommand{\L}{\Lambda}

%Mathcal Letters
\newcommand{\cL}{\mathcal{L}}
\newcommand{\cT}{\mathcal{T}}
\newcommand{\cA}{\mathcal{A}}
\newcommand{\cW}{\mathcal{W}}

%Mathematical operators
\newcommand{\INT}{\int_{\O}}
\newcommand{\DINT}{\int_{\d\O}}
\newcommand{\Int}{\int_{-\infty}^{\infty}}
\newcommand{\del}{\partial}

\newcommand{\inpr}[2]{\left\langle #1,#2 \right\rangle}
\newcommand{\abs}[1]{\left\lvert{#1}\right\rvert}
\newcommand{\fr}[2]{\frac{#1}{#2}}
\newcommand{\x}{\times}
\DeclareMathOperator{\Tr}{Tr}
\DeclareMathOperator{\Id}{Id}

\DeclareMathOperator{\dive}{div}
\DeclareMathOperator{\id}{id}
\DeclareMathOperator{\pr}{pr}
\DeclareMathOperator{\Diff}{Diff}
\DeclareMathOperator{\supp}{supp}
\DeclareMathOperator{\graph}{graph}
\DeclareMathOperator{\osc}{osc}
\DeclareMathOperator{\const}{const}
\DeclareMathOperator{\dist}{dist}
\DeclareMathOperator{\loc}{loc}
\DeclareMathOperator{\grad}{grad}
\DeclareMathOperator{\Ric}{Ric}
\DeclareMathOperator{\Rm}{Rm}
\DeclareMathOperator{\Sc}{R}
\DeclareMathOperator{\Ein}{G}
\DeclareMathOperator{\opEin}{\mathcal{G}}
\DeclareMathOperator{\Sch}{P}
\DeclareMathOperator{\W}{\mathcal{W}}
\DeclareMathOperator{\inj}{inj}
\DeclareMathOperator{\adj}{adj}
\DeclareMathOperator{\Sym}{Sym}

%Environments
\newcommand{\Theo}[3]{\begin{#1}\label{#2} #3 \end{#1}}
\newcommand{\pf}[1]{\begin{proof} #1 \end{proof}}
\newcommand{\eq}[1]{\begin{equation}\begin{alignedat}{2} #1 \end{alignedat}\end{equation}}
\newcommand{\IntEq}[4]{#1&#2#3	 &\quad &\text{in}~#4,}
\newcommand{\BEq}[4]{#1&#2#3	 &\quad &\text{on}~#4}
\newcommand{\br}[1]{\left(#1\right)}

%Logical symbols
\newcommand{\Ra}{\Rightarrow}
\newcommand{\ra}{\rightarrow}
\newcommand{\hra}{\hookrightarrow}
\newcommand{\mt}{\mapsto}

%Names
\newcommand{\holder}{H\"older}

%Fonts
\newcommand{\mc}{\mathcal}
\renewcommand{\it}{\textit}
\newcommand{\mrm}{\mathrm}

%Spacing
\newcommand{\hp}{\hphantom}


%\parindent 0 pt

\protected\def\ignorethis#1\endignorethis{}
\let\endignorethis\relax
\def\TOCstop{\addtocontents{toc}{\ignorethis}}
\def\TOCstart{\addtocontents{toc}{\endignorethis}}


\newcommand{\note}[1]{\Rd {\bf[[ #1 ]]} \Bk}

\DeclareMathOperator{\operatorsymbol}{\sigma}
\DeclareMathOperator{\norfactor}{\lambda}
\DeclareMathOperator{\ineq_rel}{\simeq}
\DeclareMathOperator{\chordarcprofile}{\mathcal{Z}}
\DeclareMathOperator{\bigo}{\mathcal{O}}
\DeclareMathOperator{\tang}{T}
\DeclareMathOperator{\nor}{N}
\newcommand{\avg}[1]{\overline{{#1}}}
\DeclareMathOperator{\sfn}{u}

\title[Chord arc comparison for preserving flows]{Chord arc comparison for curve flows preserving a global functional}

\date{}

\dedicatory{}
\subjclass[2010]{58J35, 35K10, 58B20}
\keywords{curves, curve shortening, distance, comparison}

\begin{document}

\begin{abstract}
Flows of plane curves that preserve a global geometric quantity such as total length and enclosed area are studied via distance comparison techniques.
\end{abstract}

\maketitle

\section{Introduction}
\label{sec:intro}

We consider closed curves in the plane evolving by the flow
\begin{equation}
\label{eq:flow}
\tag{FLOW}
\partial_t X = -(\kappa - h(\gamma_t)) \nor
\end{equation}
where \(X : \S^1 \times [0, T)\) is a one-parameter family of immersions with unit normal field \(\nor\), curvature \(\kappa\), \(\gamma_t = \gamma(\S^1, 2)\) is the image of \(X_t = X(\cdot, t)\).

The term \(h\) is a global term. That is,
\[
h : C^{\infty} (\S^1 \to \R^2) \to \R
\]
is a functional taking curves to real numbers. For geometric quantities, since the only pointwise invariant is the curvature \(\kappa\), \(h\) will typically be an integral of \(\kappa\) of the form \(\int_{\S^1} f(\kappa) ds\) for some smooth function \(f\). But it may also involve geometrically invariant integrals such as of the support function \(\sfn = \inpr{X}{\nor}\) as in the total enclosed area \(A = \tfrac{1}{2} \int_{\S^1} \sfn ds\). We may also allow \(h\) to be \emph{anisotropic} so that \(h = \int_{\S^1} f(\kappa, X, \nor) ds\) and in particular, \(h = \int_{\S^1} f(\kappa, \sfn) ds\).

Specific examples include the \emph{curve shortening flow}, given by
\begin{equation}
\label{eq:csf}
\tag{CSF}
h \equiv 0,
\end{equation}
and the \emph{length preserving curve shortening flow}, given by
\begin{equation}
\label{eq:lpcsf}
\tag{LPCSF}
h = \frac{1}{2\pi} \int_{\S^1} \kappa^2 ds,
\end{equation}
and the \emph{enclosed area preserving curve shortening flow} given by
\begin{equation}
\label{eq:apcsf}
\tag{APCSF}
h = \frac{2\pi}{L} = \frac{2\pi}{\int_{\S^1} ds}.
\end{equation}

\section{Basic conventions, quantities and evolution equations}
\label{sec:basic}

We take the convention that \(\tang\) is the unit tangent and \(\nor\) is the \emph{outer} unit normal. To ensure convex curves have positive curvature we then use the convention for the Frenet-Serret equations,
\begin{equation}
\label{eq:fs}
\partial_s \tang = - \kappa \nor, \quad \partial_s \nor = \kappa \tang.
\end{equation}

Let us define \(v = \abs{X'}\) where the prime denotes differentiation with respect to the \(\S^1\) variable commuting with \(t\) in the parametrisation \(X : \S^1 \times [0, T) \to \R^2\). Then the evolution equations are derived from the evolution of \(v\) and the equations defining the various quantities. We record them in the following lemma.

\begin{lemma}
\label{lem:basic_evolution}
The following evolution equations hold:
\begin{enumerate}
\item \(\partial_t v = -k(k-h) v\)
\item \([\partial_t, \partial_s] = \kappa(\kappa - h) \partial_s\)
\item \(\partial_t ds = -\kappa(\kappa - h) ds\)
\item \(\partial_t \tang = -\partial_s \kappa\nor\)
\item \(\partial_t \nor = \partial_s \kappa \tang\)
\item \(\partial_t \kappa = \partial_s^2 \kappa + \kappa^2(\kappa - h)\).
\end{enumerate}
\end{lemma}

Let us now state the definitions we use for intrinsic and extrinsic distances. Note in particular, the intrinsic distance we use is oriented which is for convenience and makes for slightly cleaner exposition later.

\begin{defn}
\label{defn:dist}
Let \((p,q) \in \gamma \times \gamma\). The \emph{extrinsic distance} is defined to be,
\[
d_t(p, q) = \|\gamma_t(q) - \gamma_t(p)\|.
\]
The \emph{oriented intrinsic distance} or \emph{oriented arc length} is defined to be,
\[
\ell_t(p, q) = \int_p^q ds_t.
\]
\end{defn}

Note that by oriented, we mean the length along \(\gamma\) from \(p\) to \(q\) in the chosen orientation of \(\gamma\) and \(\ell(p, q)\) is always non-negative. Note also that with our definition \(\ell\) is not symmetric but satisfies,
\begin{equation}
\label{eq:ell_symmetry}
\ell(q, p) = 2\pi L - \ell(p, q).
\end{equation}
The usual intrinsic distance between \(p\) and \(q\), denoted here by \(\tilde{\ell}(p, q)\), satisfies
\[
\tilde{\ell}(p, q) = \min\{\ell(p, q), \ell(q, p)\} = \min\{\ell(p, q), 2\pi L - \ell(p, q)\}.
\]

Let us also define
\begin{equation}
\label{eq:w}
w(p, q) = \frac{\gamma(q) - \gamma(p)}{d(p,q)},
\end{equation}
the unit vector pointing from \(p\) to \(q\).

\begin{lemma}
\label{lem:distance_evolution}

\begin{align*}
\partial_t d_t(p, q) &= -(\kappa_q-h) \inpr{\nor_q}{w} + (\kappa_p-h) \inpr{\nor_p}{w} \\
\partial_t \ell_t(p, q) &= - \int_p^q \kappa(\kappa-h) ds \\
\partial_t L &= - \int_{\gamma} \kappa(\kappa-h) ds
\end{align*}
\end{lemma}

\begin{proof}
From the flow equation \eqref{eq:flow}, we obtain
\[
\begin{split}
\partial_t \inpr{\gamma(q) - \gamma(p)}{\gamma(q) - \gamma(p)} &= 2 \inpr{-(\kappa(q)-h) \nor(q)}{\gamma(q) - \gamma(p)} \\
&\quad  - 2 \inpr{-(\kappa(p)-h) \nor(p)}{\gamma(q) - \gamma(p)} \\
&= 2 d\left[-\inpr{(\kappa(q)-h)\nor(q)}{w} + \inpr{(\kappa(p)-h)\nor(p)}{w}\right]
\end{split}
\]
The evolution of \(d = \left(\inpr{\gamma(q) - \gamma(p)}{\gamma(q) - \gamma(p)}\right)^{1/2}\) follows by the chain rule.

Using the evolution of \(ds\) from Lemma \ref{lem:basic_evolution},
\[
\partial_t \ell(p,q) = \partial_t \int_p^q ds_t = -\int_p^q \kappa(\kappa-h) ds
\]
\end{proof}

\section{The chord arc profile}
\label{sec:chordarc}

In this section we introduce the chord-arc profile of a simple closed curve and derive some its properties. The results are particularly appealing for strictly convex curves. The discussion here is closely analogous to one found in \cite{Bryan} for the isoperimetric profile.

\begin{defn}
Let $\gamma$ be a smooth embedded, closed curve in the plane with total length $2\pi$ given by the embedding $X: \S^1 \to \R^2$. For $x\in [0,2\pi]$, the chord-arc profile of $\gamma$ is defined as
\[
\chordarcprofile (x) = \frac{2\pi}{L} \inf\left\{d(p, q): (p,q) \in \S^1\times\S^1, \ell(p,q) = \frac{L}{2\pi} x\right\}
\]
where $d(p,q) = \abs{X(p) - X(q)}$ is the \emph{extrinsic} distance in $\R^2$ between $p$ and $q$ and $\ell(p,q) = \int_p^q ds$ is the oriented intrinsic distance from $p$ to $q$ as in \Cref{defn:dist}. With this definition, from the symmetry of \(\ell\) in equation \eqref{eq:ell_symmetry}, we see that $\chordarcprofile$ is symmetric about $\pi$.
\end{defn}

\begin{rem}
By definition, the chord-arc profile is scale invariant and defined on a uniform interval \([0, 2\pi]\) independently of \(\gamma\). This allows us to compare the profile of different curves. The constant \(2\pi\) is chosen for convenient comparison with the unit circle. Scale invariance also helps act as a guide throughout the computations - everything we do must be scale invariant!
\end{rem}

\begin{rem}
By compactness of $\S^1\times \S^1$ (which parametrises the set of connected arcs of $\gamma$) and continuity of $\ell$, $d$, for any $x$, the infimum is attained so that there exists $(p_0,q_0) \in \S^1 \times \S^1$ with $\ell(p_0, q_0) = x$ and such that $\chordarcprofile(x) = d(p_0, q_0)$. Also, since $\ell(p,p) = 0$, for any $x\in(0,2\pi)$ we have $p_0 \ne q_0$ and so $d$ is smooth at $(p_0, q_0)$.
\end{rem}

The next proposition gives the asymptotic behaviour of the chord-arc profile near the end point $x=0$, and so by symmetry also at the end point $x=2\pi$. Note also the scale invariance.

\begin{prop}
\label{prop:asymptotics}
As $x\to 0$ the chord-arc profile satisfies
\[
\lim_{x\to 0} \frac{\chordarcprofile(x) - x}{x^3} = - \frac{1}{24}\left(\frac{L}{2\pi}\right)^2 \sup_{\S^1} \kappa^2.
\]
\end{prop}

\begin{proof}
Parametrise $\gamma$ by arc-length. As in \cite{MR2794630}, 
\[
d(s_1, s_2) = \ell(s_1,s_2) - \frac{\kappa(s_1)^2}{24} \ell(s_1,s_2)^3 + \bigo(\ell(s_1, s_2)^4)
\]
where the constant in the $\bigo$ term is bounded independently of $s_1,s_2$.

In particular, letting $s_{\max}$ be such that $\sup_s \kappa^2(s) = \kappa^2(s_{\max})$ we obtain
\[
\frac{d\left(s_{\max}, s_{\max} + \frac{L}{2\pi} x\right) - \frac{L}{2\pi}x}{\left(\tfrac{L}{2\pi}\right)^3 x^3} = -\frac{\sup \kappa^2}{24} + \bigo(x^4)
\]
since $\ell\left(s_{\max}, s_{\max} + \frac{L}{2\pi} x\right) = \abs{s_{\max} + \frac{L}{2\pi} x - s_{\max}} = \frac{L}{2\pi} x$.

Now let $(s_1(x), s_2(x))$ be such that $\chordarcprofile(x) = \frac{2\pi}{L} d(s_1(x), s_2(x))$ and $\ell(s_1(x), s_2(x)) = \frac{L}{2\pi} x$. Then $\ell(\left(s_{\max}, s_{\max} + \tfrac{L}{2\pi} x\right) = \ell(s_1(x), s_2(x))$ and hence
\[
\begin{split}
0 &\leq \frac{2\pi}{L}d\left(s_{\max}, s_{\max} + \frac{L}{2\pi}x\right) - \chordarcprofile(x) \\
&= \frac{2\pi}{L}d\left(s_{\max}, s_{\max} + \frac{L}{2\pi}x\right) - \frac{2\pi}{L} d(s_1(x), s_2(x)) \\
&= \left(\frac{L}{2\pi}\right)^2 \frac{\kappa(s_1(x))^2 - \sup \kappa^2}{24} x^3 + \bigo(x^4).
\end{split}
\]
Since $\kappa(s_1(x))^2 < \sup \kappa^2$, the inequality ensures that $\kappa(s_1(x))^2 \to \sup \kappa^2$ as $x\to 0$ which gives the proposition.
\end{proof}

The main techniques we employ in this paper to understand the chord-arc profile are variational and we begin with the first and second variations of the geometric quantities $d$ and $\ell$. The derivation is fairly standard these days, but we give the full details here to be complete. It will prove useful to make various definitions that will feature throughout this paper.

Let $\theta(p,q)$ denote the angle between $\tang_{p}$ and $w = (1/d)(X(q) - X(p))$ so that $\inpr{w}{\tang_p} = \cos\theta$. Define the following "indicator" functions,
\begin{align*}
\xi(p, q) &= \begin{cases}
0,& \tang_{p} \ne \tang_{q} \\
1,& \tang_{p} = \tang_{q}
\end{cases} \\
\intertext{and}
\delta(p, q) &= \begin{cases}
0,& \inpr{w}{\nor_{p}} < 0 \\
1,& \inpr{w}{\nor_{p}} > 0.
\end{cases}
\end{align*}
Also define the matrices
\begin{align*}
K(p,q) &= (-1)^{\delta+1}\begin{pmatrix}
\kappa_{p} & 0 \\
0 & (-1)^{\xi} \kappa_{q}
\end{pmatrix} \\
M_{\pm} &= \begin{pmatrix}
1 & \pm 1 \\
\pm 1 & 1 
\end{pmatrix}
\end{align*}
and
\[
M(p,q) = \begin{cases}
M_+,& \xi(p,q) = 0 \\
M_-,& \xi(p,q) = 1.
\end{cases}
\]
We think of $K$, $M_+$ and $M_-$ as linear operators acting on $T(\S^1\times\S^1)$ written with respect to the basis $\{\tang_p, \tang_q\}$. Observe that $M_+$ and $M_-$ are simultaneously diagonalised by the eigenvectors 
\begin{equation}
\label{eq:eigenvectors}
v_1 = \tang_{p} + \tang_{q}, \quad v_2 = \tang_{p} - \tang_{q}
\end{equation}
with corresponding eigenvalues
\begin{equation}
\label{eq:eigenvalues}
\begin{split}
\lambda^+_1 &= \lambda^-_2 = 2 \\
\lambda^+_2 &= \lambda^-_1 = 0.
\end{split}
\end{equation} 

\begin{rem}
\label{rem:convex}
In the case of strictly convex curves \(\gamma\), the unit tangent map \(\tang : \S^1 \to \S^1\) is a diffeomorphism so that \(\tang_p \ne \tang_q\) for \(p \ne q\). Thus \(\xi = 0\). Moreover, since \(\gamma\) encloses a convex body and the line \(\gamma(p) + t w(p, q)\) with \(0 \leq t \leq d(p, q)\) joins \(\gamma(p)\) to \(\gamma(q)\) the entire line is contained in the region enclosed by \(\gamma\). Therefore \(w\) points inward and since \(\nor\) is the outer unit normal, \(\inpr{w}{N_p} < 0\) and \(\delta = 0\). Thus in the convex case,
\[
K = -\begin{pmatrix}
\kappa_{p} & 0 \\
0 & \kappa_{q}
\end{pmatrix}
\]
is negative definite and
\[
M = M_+ = \begin{pmatrix}
1 & 1 \\
1 & 1
\end{pmatrix}.
\]
\end{rem}

\begin{prop}[Spatial Variation]
\label{prop:spatial_var}
Let $(p_0, q_0) \in \S^1 \times \S^1$ be such that $d(p_0, q_0)$ is minimum amongst all $(p,q)$ such that $\ell(p, q) = \ell(p_0, q_0)$ (or more briefly, $\chordarcprofile(\ell(p_0,q_0)) = d(p_0, q_0)$), then 
\[
D^2d_{(p_0,q_0)} =  \sqrt{1-\cos^2\theta}K + \frac{1-\cos^2 \theta}{d} M.
\]
\end{prop}

\begin{proof}
Let us begin by deriving variational formulae for $d$ and $\ell$ without constraint. Let $\omega_p$ and $\omega_q$ denote the $1$-forms dual to the unit tangent vectors $\tang_p$ and $\tang_q$ at $p$ and $q$ respectively. Parametrise $\gamma$ by arc-length, so that $\partial_p \ell = - 1$ and $\partial_q \ell = 1$ which gives 
\begin{equation}
\label{eq:l_first_var}
D\ell = -\omega_p + \omega_q.
\end{equation}

Next, differentiating $d$ and using $\partial_pX = \tang_p$, $\partial_qX = \tang_q$, the unit tangents at $p$ and $q$ respectively, gives
\[
\partial_p d (p, q) = \frac{1}{d} \inpr{\partial_p(X(q) - X(p))}{X(q) - X(p)} = -\inpr{w}{\tang_p}
\]
and similarly for $\partial_q d$ (but with the sign changed) so that
\begin{equation}
\label{eq:d_first_var}
Dd = -\inpr{w}{\tang_p} \omega_p + \inpr{w}{\tang_q} \omega_q.
\end{equation}

For the second variation, we first differentiate $w$:
\[
\begin{split}
\partial_p w &= \partial_p \frac{1}{d} (X(q) - X(p)) \\
&= \frac{1}{d^2} \inpr{w}{\tang_p} (X(q) - X(p)) - \frac{1}{d} \tang_p \\
&= -\frac{1}{d}\left(\tang_p - \inpr{w}{\tang_p} w\right).
\end{split}
\]
and similarly for $\partial_q w$ (again with the sign changed).  Putting this together with the Frenet-Serret formula $\partial_p T_p = -\kappa_p \nor_p$ in equation \eqref{eq:fs} then gives
\[
d_{pp} = \inpr{\frac{1}{d}\left(\tang_p - \inpr{w}{\tang_p} w\right)}{\tang_p} + \inpr{w}{\kappa_p \nor_p} = \frac{1}{d}\left(1 - \inpr{w}{\tang_p}^2\right) + \inpr{w}{\kappa_p \nor_p}.
\]
Similar computations give $d_{qq}$ and $d_{pq}$ leading to
\begin{equation}
\label{eq:d_second_var}
\begin{split}
D^2d &= \left(\inpr{w}{\kappa_p \nor_p} + \frac{1}{d}(1 - \inpr{w}{\tang_p}^2) \right) \omega_p \otimes \omega_p \\
&+ \left(-\inpr{w}{\kappa_q \nor_q} + \frac{1}{d}(1 - \inpr{w}{\tang_q}^2) \right) \omega_q \otimes \omega_q \\
&- \left(\frac{1}{d}\left(\inpr{\tang_p}{\tang_q} - \inpr{w}{\tang_p}\inpr{w}{\tang_q}\right) \right) \left(\omega_p \otimes \omega_q + \omega_q \otimes \omega_p\right).
\end{split}
\end{equation}

Now, consider the curve $\alpha: u \mapsto (p_0,q_0) + uv_1 = (p_0,q_0) + (u, u) \in \S^1\times \S^1$ which satisfies,
\[
\partial_u \ell(\alpha(u)) = 0 
\]
by equation \eqref{eq:l_first_var}. Thus $\ell$ is constant, equal to $\ell(p_0, q_0)$ along the curve $\alpha$. 

Since $(p_0, q_0)$ minimises $d$ amongst all $(p,q)$ with $\ell(p, q) = \ell(p_0, q_0)$, we have that $u=0$ is a local minima of the function $u\mapsto d(\alpha(u))$. Then from the first variation of $d$ \eqref{eq:d_first_var},
\[
0 = \left.\partial_u\right|_{u=0} d(\alpha(u)) = -\inpr{w}{\tang_p} + \inpr{w}{\tang_q}
\]
so that
\[
\cos\theta = \inpr{\tang_p}{w} = \inpr{\tang_q}{w}.
\]
Thus either $\tang_{p_0} = \tang_{q_0}$, or $\tang_{p_0} \ne \tang_{q_0}$ and $w$ bisects $\tang_{p_0}$ and $\tang_{q_0}$. These cases are recorded by the indicator function $\xi$ equal to \(1\) in the first case and \(0\) in the second. Moreover we have $\inpr{w}{\nor_{p_0}} = \inpr{w}{\nor_{q_0}}$ if \(\xi = 1\) and $\inpr{w}{\nor_{p_0}} = -\inpr{w}{\nor_{q_0}}$ if \(\xi=0\) so that in either case, $\inpr{w}{\nor_{p_0}} = (-1)^{\xi + 1} \inpr{w}{\nor_{q_0}}$. We can also deduce that $\inpr{w}{\nor_{p_0}} = \pm \sqrt{1-\cos^2\theta}$ with the sign depending on whether $w$ points into or out of $\gamma_{t_0}$ at $p_0$ which we can then write as
\[
\inpr{w}{\nor_{p_0}} = (-1)^{\delta+1} \sqrt{1-\cos^2\theta}.
\]
Succinctly, we have
\begin{equation}
\label{eq:Nw}
(-1)^{\xi+1} \inpr{w}{\nor_{q_0}} = \inpr{w}{\nor_{p_0}} =  (-1)^{\delta+1} \sqrt{1-\cos^2\theta}.
\end{equation}

\emph{Case 1}: $\tang_{p_0} \ne \tang_{q_0}$ ($\xi = 0$).

Since $w$ bisects $\tang_{p_0}$ and $\tang_{q_0}$, applying the double angle formula, we also have 
\[
\inpr{\tang_{p_0}}{\tang_{q_0}} = 2\cos^2\theta - 1.
\]
Substituting this expression and equation \eqref{eq:Nw} into the second variation of $d$ \eqref{eq:d_second_var} gives
\begin{align*}
D^2d_{(p_0,q_0)} &=  \left((-1)^{\delta+1} \sqrt{1-\cos^2\theta}\kappa_{p_0} + \frac{1}{d}(1 - \cos^2\theta) \right) \omega_{p_0} \otimes \omega_{p_0} \\
&+ \left((-1)^{\delta+1} \sqrt{1-\cos^2\theta} \kappa_{q_0} + \frac{1}{d}(1 - \cos^2\theta) \right) \omega_{q_0} \otimes \omega_{q_0} \\
&+ \left(\frac{1}{d}(1 - \cos^2\theta) \right) \left(\omega_{p_0} \otimes \omega_{q_0} + \omega_{q_0} \otimes \omega_{p_0} \right)
\end{align*}
which gives the desired expression for $\xi=0$ when expressed in matrix form.

\emph{Case 2}: $\tang_{p_0} = \tang_{q_0}$ ($\xi = 1$).

In this case, we have $\inpr{\tang_{p_0}}{\tang_{q_0}} = 1$. Substituting this and equation \eqref{eq:Nw} into the second variation of $d$ \eqref{eq:d_second_var} gives
\begin{align*}
D^2d_{(p_0,q_0)} &=  \left((-1)^{\delta+1} \sqrt{1-\cos^2\theta}\kappa_{p_0} + \frac{1}{d}(1 - \cos^2\theta) \right) \omega_{p_0} \otimes \omega_{p_0} \\
&+ \left(-(-1)^{\delta+1} \sqrt{1-\cos^2\theta} \kappa_{q_0} + \frac{1}{d}(1 - \cos^2\theta) \right) \omega_{q_0} \otimes \omega_{q_0} \\
&- \left(\frac{1}{d}(1 - \cos^2\theta) \right) \left(\omega_{p_0} \otimes \omega_{q_0} + \omega_{q_0} \otimes \omega_{p_0} \right)
\end{align*}
which gives the desired expression for $\xi=1$ when expressed in matrix form.

\end{proof}

The next lemma gives a useful first application of the first variation formula for $d$ that allows us to exploit concavity properties of the chord-arc profile. 

\begin{lemma}
\label{lem:concave_barrier}
If there exists a strictly concave, positive function $\phi: (0, 2\pi) \to \R$ that is symmetric about $\pi$ and supporting $\chordarcprofile$ at $x_0= \tfrac{2\pi}{L} \ell(p_0,q_0)$ (so that $\phi(x) \leq \chordarcprofile(x)$ and $\phi(x_0)=\chordarcprofile(x_0)$), then $\tang_{p_0} \ne \tang_{q_0}$.
\end{lemma}

\begin{rem}
In particular, if $\gamma$ is strictly convex, then already \(T=\tang_{p_0} \ne \tang_{q_0}\) whenever \(p_0 \ne q_0\). The lemma then says that the in the presence of a concave lower barrier, we obtain the same conclusion and we see this \emph{local concavity} of the profile gives convex like behaviour of the corresponding point. \Cref{thm:barrier} below gives a converse where convexity of the curve implies concavity of the profile. Thus we see a close relationship between convexity of the curve and concavity of the profile (even in a local sense).
\end{rem}

\begin{proof}
Suppose there is a $\phi$ as in the statement of the lemma. We proceed as described in \cite{MR2794630}. To obtain a contradiction, let us suppose that $\tang_{p_0} = \tang_{q_0} \ne w$. Then the normal makes an acute angle with the chord $\overline{p_0 q_0}$ at one endpoint, and an obtuse angle at the other. Therefore points on the chord near one endpoint are inside the region enclosed by $\gamma$, while points near the other endpoint are outside, implying that there is at least one other point where the curve $\gamma$ meets the chord. We may assume that an intersection occurs at $s$ with $p_0 < s < q_0$. Then we have
\begin{align*}
d(p_0, q_0) &= d(p_0, s) + d(s, q_0) \\
\ell(p_0, q_0) &= \min\{\ell(p_0, s) + \ell(s, q_0),  2\pi -\ell(p_0,s ) - \ell(s, q_0)\}.
\end{align*}
Since $\phi$ is strictly concave, $\phi(x + y) = \phi(x + y) + \phi(0) \leq \phi(x) + \phi(y)$ whenever $x, y > 0$ and $x + y < 2\pi$. Writing
\[
x_0 = \tfrac{2\pi}{L}\ell(p_0, q_0), \quad x_1 = \tfrac{2\pi}{L}\ell(p_0, s), \quad x_2 = \tfrac{2\pi}{L}\ell(s, q_0)
\]
and noting also that $\phi(x) = \phi(2\pi - x)$, we have
\[
\begin{split}
0 &= \chordarcprofile(x_0) - \phi(x_0) =  \frac{2\pi}{L} d(p_0, q_0) - \phi(x_0) \\
&= \frac{2\pi}{L}d(p_0, s) + \frac{2\pi}{L} d(s, q_0) - \phi(x_1 + x_2) \\
&> \frac{2\pi}{L} d(p_0, s) - \phi(x_1) + \frac{2\pi}{L}d(s, q_0) - \phi(x_2) \\
&\geq \chordarcprofile(x_1) - \phi(x_1) + \chordarcprofile(x_2) - \phi(x_2) \\
&\geq 0.
\end{split}
\]
Thus we have a contradiction. 
\end{proof}

Next, from these variational formulae, we obtain a (weak) differential inequality for $\chordarcprofile$.

\begin{thm}
\label{thm:barrier}
The chord-arc profile $\chordarcprofile$ satisfies the following differential inequality in the support (or barrier, or sometimes Calabi) sense
\begin{align*}
Z'_- &\leq \cos\theta \leq Z'_+, \\
Z'' &\leq \frac{L}{2\pi}\frac{(-1)^{\delta+1}\sqrt{1-(Z')^2}}{4} (\kappa_{p_0} + (-1)^{\xi}\kappa_{q_0}) + \xi \frac{1-(Z')^2}{Z}.
\end{align*}
In particular, if $\gamma$ is convex, then
\[
Z'' \leq - \frac{L}{2\pi} \frac{\sqrt{1-(Z')^2}}{4} (\kappa_{p_0} + \kappa_{q_0}) \leq 0
\]
and hence $\chordarcprofile$ is concave. If $\gamma$ is strictly convex, then $\chordarcprofile$ is strictly concave.
\end{thm}

Recall that the support inequality means for every $x_0 \in [0,2\pi]$ there exists a smooth function $Z$ defined in a neighbourhood of $x_0$ such that $Z(x) \geq \chordarcprofile$, $Z(x_0) = \chordarcprofile(x_0)$ and
\begin{align*}
Z' &= \cos \theta, \\
Z'' &= \frac{(-1)^{\delta+1}\sqrt{1-(Z')^2}}{4} (\kappa_{p_0} + (-1)^{\xi}\kappa_{q_0}) + \xi \frac{1-(Z')^2}{Z}.
\end{align*}

\begin{proof}
For any $x_0$, again let $(p_0,q_0)$ be such that $\frac{2\pi}{L}\ell(p_0,q_0) = x_0$ and $\chordarcprofile(x_0) = \frac{2\pi}{L} d(p_0,q_0)$. Consider the curve
\[
\alpha(u) = (p_0, q_0) + uv_2 = (p_0, q_0) + (u,-u).
\]
This satisfies
\[
\partial_u \ell(\alpha(u)) = -2
\]
so that $\ell\circ\alpha$ has a smooth local inverse near $u=0$ so that \(\ell \circ \alpha(0) = \frac{L}{2\pi} x_0\) and \((\ell \circ \alpha)^{-1} \left(\frac{L}{2\pi} x_0\right) = 0\). Thus we can define the smooth function
\[
Z(x) = \frac{2\pi}{L} d\left(\alpha \circ (\ell\circ\alpha)^{-1} \left(\frac{L}{2\pi}x\right)\right)
\]
which is an upper supporting function for \(\chordarcprofile\) at \(x_0\):
\begin{align*}
Z(x_0) &= \frac{2\pi}{L} d(p_0, q_0) = \chordarcprofile(x_0) \\
\intertext{and}
Z(x) &= \frac{2\pi}{L} d\left(\alpha \circ (\ell\circ\alpha)^{-1} \left(\frac{L}{2\pi}x\right)\right) \geq \chordarcprofile \left((\ell\circ\alpha)^{-1} \left(\frac{2\pi}{L}x\right)\right) = \chordarcprofile(x).
\end{align*}

Observe that $\alpha' = v_2$ and
\[
\partial_x \left[(\ell\circ\alpha)^{-1} \left(\frac{L}{2\pi}x\right)\right] = -\frac{1}{2} \frac{L}{2\pi}.
\]
Therefore, using also the first variation of $d$ \eqref{eq:d_first_var}, we have
\[
Z'(x_0) = \frac{2\pi}{L} D d \cdot \left(-\frac{1}{2}\frac{L}{2\pi} v_2\right) = -\frac{1}{2} \cos\theta (-\omega_{p_0} + \omega_{q_0}) \cdot (\tang_{p_0} - \tang_{q_0}) = \cos\theta
\]
proving the first equation.

For the second equation, we have (as a quadratic form)
\[
Z''(x_0) = \frac{2\pi}{L} D^2 d \left(-\frac{1}{2}\frac{L}{2\pi} v_2\right)
\]
Now apply Proposition \ref{prop:spatial_var} to obtain
\[
\begin{split}
Z''(x_0) &= \frac{L}{8\pi} \left[\sqrt{1-\cos^2\theta} K (v_2) + \frac{1-\cos^2\theta}{d} M (v_2)\right] \\
&= \frac{L}{2\pi} \frac{\sqrt{1-(Z')^2}}{4} K (v_2) + \frac{1-(Z')^2}{4Z} M (v_2).
\end{split}
\]
For the first term we have
\begin{align*}
K (v_2) &= (-1)^{\delta+1}
\begin{pmatrix}
1 & -1
\end{pmatrix}
\begin{pmatrix}
\kappa_{p_0} & 0 \\
0 & (-1)^{\xi} \kappa_{q_0}
\end{pmatrix}
\begin{pmatrix}
1 \\
-1
\end{pmatrix} \\
&= (-1)^{\delta+1} (\kappa_{p_0} + (-1)^{\xi} \kappa_{q_0})
\end{align*}
which gives the first term in the second equation of the proposition.

For the second term, we treat the two cases $\xi=1$ and $\xi=0$ separately. In the first case, we just observe that $v_2$ is a null eigenvector of $M=M_+$ so that the second term vanishes when $\xi=0$. 

For case two, we have $M=M_-$ and we can't apply the same trick to kill the second term, since this would require we use the null eigenvector $v_1$ of $M_-$ in the definition of the curve $\alpha$. But recall that from the proof of Proposition \ref{prop:spatial_var}, $v_1$ is annihilated by $d\ell$ and so we can't invert $\ell\circ\alpha$ in this situation. Instead, we make do again with using $v_2$, which has length $\sqrt{2}$ and is an eigenvector of $M_-$ with eigenvalue $2$ so that $M_- (v_2) = 4$ which leads to the required second term when $\xi=1$.

Finally, if $\gamma$ is convex, then $\kappa \geq 0$ and $\tang_{p_0} \ne \tang_{q_0}$ so that $\xi = 1$. Moreover, since the closure of the region bounded by $\kappa$ is a convex body, and since $X(p_0), X(q_0)$ lie in the closure of this region, for any $u \in [0,1]$, we must have that $X(p_0) + u(X(q_0) - X(p_0))$ also lies in this region so that $(X(q_0) - X(p_0))$ (and hence $w = \tfrac{1}{d(p_0,q_0)} (X(q_0) - X(p_0))$) points inward ensuring that $\delta = 1$. Therefore, at every point $x_0$, the function $Z$ satisfies
\[
Z'' \leq - \frac{L}{2\pi} \frac{\sqrt{1-(Z')^2}}{4} (\kappa_{p_0} + \kappa_{q_0}) \leq 0
\]
and $\chordarcprofile$ is everywhere supported above by a concave function, hence is itself concave \cite{MR1674097}.
\end{proof}

\begin{rem}
Compare the above with similar results obtained in \cite{MR1674097}, pertaining to isoperimetric regions of convex bodies in Euclidean space and in \cite{MR875084}, pertaining to isoperimetric regions in compact surfaces. As described in \cite{pbthesis}, both arguments lead to the concavity of $I^2(x) + K_0 x^2$ where $I$ is the isoperimetric profile, and $K_0$ is a lower bound on boundary mean curvature or ambient Gauss curvature respectively. In particular, if the curvature is non-negative, then not only is the isoperimetric profile concave, but it's square also. Here, the analogous result is not true for the chord-arc profile in the strongest possible sense as seen by the simple counter-example of the unit circle, whose chord-arc profile is 
\[
\chordarcprofile_{\S^1} (x) = 2 \sin \left(\frac{x}{2}\right).
\]
This is a strictly concave function whose square is not concave. This seems related to the fact that the local behaviour of the chord-arc profile is determined by $\kappa^2$ as opposed to the isoperimetric profile of a surface or convex body in the plane, which is locally determined by the Gauss curvature or boundary mean curvature as appropriate. The difference is that for the chord-arc profile the sign of the curvature is irrelevant.
\end{rem}

\section{The evolving chord arc profile}
\label{sec:evolving_chordarc}

The chord-arc profile is scale invariant, but for curves evolving by the curve shortening flow, we must also account for the time scaling.

\begin{defn}
\label{defn:timeprofile}
Let \(\gamma_t\) be a smooth, one-parameter family of curves defined on the interval \(t \in [0, T)\) and evolving by the curve shortening flow. The \emph{evolving} chord-arc profile is defined to be
\[
\chordarcprofile(x, \tau) = \chordarcprofile_{\gamma_{t(\tau)}} (x)
\]
where
\[
\begin{cases}
\partial_t \tau &= \left(\frac{2\pi}{L}\right)^2 \\
\tau(0) &= 0.
\end{cases}
\]
\end{defn}

\begin{thm}
\label{thm:viscosity}
Let $X: \S^1\times [0,T)$ be a solution of the flow \eqref{eq:flow}. For any \(x_0, t_0\), let $(p_0,q_0)$ be points such that $\tfrac{2\pi}{L(t_0)} d(p_0, q_0, t_0) = \chordarcprofile(x_0)$ and $x_0 = \frac{2\pi}{L(t_0)} \ell(p_0, q_0, t_0)$. The chord-arc profile is a viscosity super-solution of the following equation
\[
\begin{split}
\partial_{\tau} Z & \geq 4 Z'' + \left(\frac{L}{2\pi}\right)^2  \avg{\kappa^2} (Z - x Z') + \frac{L}{2\pi} \int_{p_0}^{q_0} \kappa^2 ds Z' - 4\xi \frac{1 - (Z')^2}{Z} \\
&\quad - \left[(-1)^{\delta+1} \sqrt{1-(Z')^2}((-1)^{\xi+1} + 1) + \avg{\kappa^2} (Z - x Z') + 2 Z'\arccos(Z')\right] \frac{Lh}{2\pi} .
\end{split}
\]
\end{thm}

Recall that this means that for any $(x_0,\tau_0)$, if $\varphi$ is a smooth function defined in a neighbourhood of $(x_0,\tau_0)$ such that
\begin{enumerate}
\item $\varphi(x_0, \tau_0) = \chordarcprofile (x_0, \tau_0)$
\item $\varphi(x, \tau) \leq \chordarcprofile (x, \tau)$, $\tau \leq \tau_0$
\end{enumerate}
then $\varphi$ satisfies the inequality in the usual sense. Note that although we require $\varphi$ to support $\chordarcprofile$ for $x$ in a full neighbourhood of $x_0$, we only require $\tau$ to be in a "half-neighbourhood", i.e. $\tau\leq \tau_0$.

\begin{proof}
Fix $(x_0, \tau_0)$ and let \(t_0 = t(\tau_0)\). Let $\varphi(x,\tau)$ be a smooth function defined in an open neighbourhood of $(x_0,\tau_0)$ and such that $\varphi(x_0, \tau_0) = \chordarcprofile(x_0, \tau_0)$ and $\varphi(x, \tau) \leq \chordarcprofile(x, \tau)$ for $\tau\leq \tau_0$.

Then we have
\[
\frac{2\pi}{L(t(\tau))} d(p, q, t(\tau)) \geq \chordarcprofile\left(\frac{2\pi}{L(t(\tau))}\ell(p, q, t(\tau)), \tau\right) \geq \varphi\left(\frac{2\pi}{L(t(\tau))}\ell(p, q, t(\tau)), \tau\right)
\]
for $\tau\leq \tau_0$ and all $(p,q,\tau)$ in a neighbourhood of $(p_0,q_0,\tau_0)$ on which $\varphi$ is defined for all $x=\frac{2\pi}{L(t(\tau))} \ell(p,q,t(\tau))$. Moreover, we have equality at $(p_0, q_0, \tau_0)$ so that the smooth function
\[
\Phi(p, q, \tau) = \tfrac{2\pi}{L(t(\tau))}d(p,q,t(\tau)) - \varphi\left(\frac{2\pi}{L(t(\tau))}\ell(p, q, t(\tau)), \tau\right)
\]
satisfies $\partial_{\tau} \Phi \leq 0$ at $(p_0, q_0, \tau_0)$. That is,
\begin{equation}
\label{eq:tau_ineq}
\begin{split}
0 &\geq \frac{\partial t}{\partial \tau} \partial_t \left[\frac{2\pi}{L} d\right] - \frac{\partial t}{\partial \tau} \varphi' \partial_t \left[\frac{2\pi}{L}\ell\right] - \partial_{\tau} \varphi \\
&= \left(\frac{L}{2\pi}\right)^2 \left(\partial_t \left[\frac{2\pi}{L} d\right] - \varphi' \partial_t \left[\frac{2\pi}{L}\ell\right]\right) - \partial_{\tau} \varphi.
\end{split}
\end{equation}

Using the time variation formula from \Cref{lem:distance_evolution}, at \((p_0, q_0, t_0)\),
\[
\begin{split}
\partial_t \left[\frac{2\pi}{L} d\right] - \varphi' \partial_t \left[\frac{2\pi}{L}\ell\right] &= \frac{2\pi}{L}\left[-(\kappa_{q_0}-h) \inpr{\nor_q}{w} + (\kappa_{p_0}-h) \inpr{\nor_p}{w}\right] + \frac{2\pi d}{L} \frac{1}{L} \int_{\gamma} \kappa(\kappa-h) ds \\
&\quad + \frac{2\pi}{L} \varphi' \int_p^q \kappa(\kappa-h) ds - \frac{2\pi\ell}{L} \frac{\varphi'}{L} \int_{\gamma}\kappa(\kappa-h) ds \\
&= \frac{2\pi}{L}\left[-(\kappa_{q_0}-h) \inpr{\nor_q}{w} + (\kappa_{p_0}-h) \inpr{\nor_p}{w}\right] \\
&\quad + \avg{\kappa(\kappa-h)} (\varphi - x \varphi') + \frac{2\pi}{L} \varphi' \int_p^q \kappa(\kappa-h) ds.
\end{split}
\]

By assumption, \(d_{t_0} (p_0, q_0)\) is minimised amongst all \((p, q)\) such that \(\ell_{t_0}(p, q) = \ell_{t_0}(p_0, q_0)\) and so we may use equation \eqref{eq:Nw} obtained for $\inpr{w}{\nor_{p_0}}$ and $\inpr{w}{\nor{q_0}}$ in the proof of \Cref{prop:spatial_var}:
\[
(-1)^{\xi+1} \inpr{w}{\nor_{q_0}} = \inpr{w}{\nor_{p_0}} =  (-1)^{\delta+1} \sqrt{1-\cos^2\theta} = (-1)^{\delta+1} \sqrt{1-(\varphi')^2}.
\]
Hence equation \eqref{eq:tau_ineq} becomes
\begin{equation}
\label{eq:time_ineq}
\begin{split}
\partial_{\tau} \varphi &\geq \frac{L}{2\pi} (-1)^{\delta+1} \sqrt{1-(\varphi')^2} [(-1)^{\xi}\kappa_{q_0} + \kappa_{p_0} - ((-1)^{\xi} + 1)h] \\
&\quad + \left(\frac{L}{2\pi}\right)^2 \avg{\kappa(\kappa-h)} (\varphi - x \varphi') + \frac{L}{2\pi} \varphi' \int_p^q \kappa(\kappa-h) ds.
\end{split}
\end{equation}

Now let \(Z\) denote the upper barrier for \(\chordarcprofile_{\gamma(t_0)}\) at \(x_0\) whose existence is guaranteed by Theorem \ref{thm:barrier}. Then we have
\[
Z(x) \geq \chordarcprofile(x, \tau_0) \geq \varphi(x, \tau_0), \quad Z(x_0) = \chordarcprofile(x_0, \tau_0) = \varphi(x_0, \tau_0).
\]
Thus \(x_0\) is a local minimum of \(Z - \varphi\) and by using Theorem \ref{thm:barrier} we have
\[
\varphi'(x_0) = Z'(x_0) = \cos\theta
\]
and
\[
\begin{split}
\varphi'' &\leq Z'' \leq \frac{L}{2\pi}\frac{(-1)^{\delta+1}\sqrt{1-(Z')^2}}{4} (\kappa_{p_0} + (-1)^{\xi}\kappa_{q_0}) + \xi \frac{1-(Z')^2}{Z} \\
&= \frac{L}{2\pi}\frac{(-1)^{\delta+1}\sqrt{1-(\varphi')^2}}{4} (\kappa_{p_0} + (-1)^{\xi}\kappa_{q_0}) + \xi \frac{1-(\varphi')^2}{\varphi}.
\end{split}
\]
Substituting into equation \eqref{eq:time_ineq} gives
\begin{equation}
\label{eq:spacetime_ineq}
\begin{split}
\partial_{\tau} \varphi &\geq 4\varphi'' + (-1)^{\delta} \sqrt{1-(\varphi')^2} [(-1)^{\xi} + 1]\frac{Lh}{2\pi} - 4\xi \frac{1-(\varphi')^2}{\varphi} \\
&\quad + \left(\frac{L}{2\pi}\right)^2 \avg{\kappa(\kappa-h)} (\varphi - x \varphi') + \frac{L}{2\pi} \varphi' \int_p^q \kappa(\kappa-h) ds.
\end{split}
\end{equation}

For the global curvature integral,
\[
\avg{\kappa(\kappa-h)} = \frac{1}{L} \int_{\gamma} \kappa^2 - h \kappa ds = \avg{\kappa^2} - \frac{2\pi}{L} h
\]
since \(h\) is independent of \(s\) and the total curvature of a simple, closed plane curve is \(2\pi\).

For the other curvature integral
\[
\int_p^q \kappa(\kappa-h) = \int_p^q \kappa^2 ds - 2 h \theta = \int_p^q \kappa^2 - 2h\arccos(\varphi')
\]
since \(w\) bisects \(\tang_p\) and \(\tang_q\) hence the angle between \(\tang_p\) and \(\tang_q\) is \(2\theta\).

Substitution into equation \eqref{eq:spacetime_ineq} completes the proof.
\end{proof}

\section{Comparison Theorem}
\label{sec:comparison}

\begin{thm}
\label{thm:comparison}
Let \(\varphi : [0, 2\pi] \times [0, \infty) \to \R\) be a continuous, strictly positive and \(C^{\infty}\) on \((0, 2\pi) \times (0, \infty)\) function that is strictly concave, symmetric about \(\pi\) with \(\varphi(0, \tau) = \varphi(2\pi, \tau) = 0\) for all \(\tau \geq 0\) with asymptotic behaviour
\[
\varphi(x, \tau) = x - C(\tau) x^3 + \bigo(x^4)  \quad \text{as} \quad x \to 0
\]
and such that
\[
\partial_{\tau} \varphi - 4 \varphi'' - \left(\frac{L}{2\pi}\right)^2 (\varphi - x \varphi') \avg{\kappa^2} - \varphi' \frac{L}{2\pi} \int_{p_0}^{q_0} \kappa^2 ds \leq 0.
\]

Then if \(\varphi (x, 0) \leq \chordarcprofile(x, 0)\) for all \(x \in (0, 2\pi)\) then \(\varphi (x, \tau) \leq \chordarcprofile(x, \tau)\) for all \(x \in (0, 2\pi)\) and \(\tau \geq 0\). Moreover
\[
\sup_{p \in \S^1} \kappa(p, t) \leq 24\left(\frac{2\pi}{L(t)}\right)^2 C(\tau(t)).
\]
\end{thm}

\begin{proof}
For convenience, let us define the \emph{linear} operator
\[
\mathcal{L} (\varphi) = 4 \varphi'' + \left(\frac{L}{2\pi}\right)^2 (\varphi - x \varphi') \avg{\kappa^2} + \varphi' \frac{L}{2\pi} \int_{p_0}^{q_0} \kappa^2 ds
\]
so the assumption of the theorem is
\[
(\partial_t - \mathcal{L}) \varphi \geq 0.
\]

We will show that for each \(\bar{\tau} > 0\), the theorem is true on \([0, \bar{\tau}]\). For \(\epsilon > 0\), define
\[
\varphi_{\epsilon} (x, \tau) = (1-\epsilon) \varphi(x, \tau) + \epsilon^2 e^{-\lambda\tau} x(2\pi - x).
\]
where \(\lambda > 0\) will be chosen momentarily but depending only on \(\bar{\tau}\). To prove that \(\varphi \leq \chordarcprofile\), it suffices to show that there exists an \(\epsilon_0 > 0\) such that for all \(\epsilon \in (0, \epsilon_0)\), \(\varphi_{\epsilon}(x, \tau) < \chordarcprofile(x, \tau)\) for all \(x \in (0, 2\pi)\) and \(\tau > 0\). Then we just take the (pointwise) limit \(\epsilon \to 0\).

Now, we have \(\varphi_{\epsilon}(0, \tau) = \varphi_{\epsilon}(2\pi, \tau) = 0\) for all \(\tau\) and \(\varphi_{\epsilon}\) is strictly concave, strictly positive for \(x \in (0, 2\pi)\) and symmetric about \(\pi\). Moreover,
\[
\varphi_{\epsilon}'(0, \tau) = (1-\epsilon) \varphi'(0, \tau) + 2\pi \epsilon^2 e^{-\lambda\tau} = 1 + \epsilon(2\pi \epsilon e^{-\lambda\tau} - 1) < 1
\]
for all \(\epsilon < 1/2\pi\) and for all \(\lambda \geq 0\). Symmetry ensures that \(\varphi_{\epsilon}'(2\pi, 0) \geq -1\) for all \(\tau\).

Since \(\chordarcprofile'(0) = -\chordarcprofile(2\pi) = 1\), we have \(\varphi_{\epsilon} (x, \tau) < \chordarcprofile(x, \tau)\) in a (time-dependent but non-empty) neighbourhood of \(\{0, 2\pi\}\). In fact, from the assumption in the theorem for the asymptotics of \(\varphi\) and Taylor's theorem with remainder, we have
\[
0 \leq \varphi_{\epsilon}(x, \tau) \leq c_1 x + c_2(\bar{\tau}) x^2.
\]
where
\[
c_1 = \left[1 + \epsilon(2\pi \epsilon - 1)\right], \quad c_2(\bar{\tau}) = \left[(1-\epsilon) \sup_{x\in [0, 2\pi], \tau \in [0, \bar{\tau}]} \varphi''\right]
\]
are upper bounds for \(\varphi_{\epsilon}'(0, \tau)\) and \(\varphi_{\epsilon}''(x, \tau)\) respectively with \(x \in [0, 2\pi]\) and \(\tau \in [0, \bar{\tau}]\). Similarly, for \(\chordarcprofile\) we have by \Cref{prop:asymptotics},
\[
\chordarcprofile(x, \tau) \geq x + A x^2 \geq 0
\]
on an interval \(x \in [0, \delta]\) and all \(\tau \in [0, \tau_0]\) where \(A \in \R\). Then
\[
\chordarcprofile(x, \tau) - \varphi_{\epsilon}(x, \tau) \geq \epsilon(1-2\pi \epsilon) x + (A - c_2(\bar{\tau})) x^2 > 0
\]
provided
\[
0 < x < \bar{x} := \min\left\{\delta, \frac{\epsilon(1-2\pi \epsilon)}{\abs{A - c_2(\bar{\tau})}}\right\}.
\]
Thus we have \(\chordarcprofile(x, \tau) > \varphi_{\epsilon}(x, \tau)\) for \((x, \tau) \in (0, \bar{x}) \times (0, \bar{\tau})\) and by symmetry also for for \((x, \tau) \in (2\pi-\bar{x}, 2\pi) \times (0, \bar{\tau})\). The important point here is that these inequalities hold true \emph{for any} \(\lambda > 0\) with \(\bar{x}\) and \(\bar{\tau}\) independent of \(\lambda\). Then anticipating the upcoming computations, we choose
\[
\lambda > \frac{\sup_{x \in [\bar{x}, 2\pi-\bar{x}]} \abs{\mathcal{L} [x(2\pi-x)]}}{\bar{x}(2\pi - \bar{x})}.
\]
depending only on \(\bar{x}\) (which only depends on \(\bar{\tau}\)).

Next, at \(\tau = 0\) and for \(x \in (0, 2\pi)\), we have
\[
\varphi_{\epsilon} (x, 0) = (1-\epsilon) \varphi(x, 0) + \epsilon^2 x(2\pi -x) < \varphi(x, 0) \leq \chordarcprofile(x, 0)
\]
for \(\epsilon \in (0, \epsilon_0) \subseteq (0, 1/2\pi)\) such that
\[
\epsilon^2 x(2\pi -x) \leq \varphi(x, 0) - (1-\epsilon) \varphi(x, 0) = \epsilon \varphi(x, 0).
\]
Such an \(\epsilon_0\) exists since
\[
\lim_{x \to 0} \frac{\varphi(x, 0)}{x (2\pi -x)} = \frac{1}{2\pi} \lim_{x \to 0} \frac{\varphi(x, 0)}{x} = \frac{1}{2\pi}
\]
and similarly for \(x \to 2\pi\) by symmetry.

Hence we have initial strict inequality and strict inequality near \(\{0, 2\pi\}\). Thus if the inequality \(\varphi_{\epsilon}(x, \tau) < \chordarcprofile(x, \tau)\) is false, there exists a first \(\tau_0 > 0\) and an interior \(x_0 \in (0, 2\pi)\) where \(\varphi_{\epsilon}(x_0, \tau_0) = \chordarcprofile(x_0, \tau_0)\). That is we have strict inequality \(\varphi_{\epsilon}(x, \tau) < \chordarcprofile(x, \tau)\) for \((x, \tau) \in (0, 2\pi) \times [0, \tau_0)\) and weak inequality \(\varphi_{\epsilon}(x, \tau_0) \leq \chordarcprofile(x, \tau_0)\) for \(x \in (0, 2\pi)\) with equality at \((x_0, \tau_0)\).

Thus at $(x_0,\tau_0)$, $\varphi_{\epsilon}$ supports \(\chordarcprofile\) from below, satisfying properties 1 and 2 in the definition of viscosity supersolutions in \Cref{thm:viscosity}. Since $\varphi_{\epsilon}$ is strictly concave, by lemma \ref{lem:concave_barrier}, $\tang_{p_0} \ne \tang_{q_0}$ and hence by \Cref{thm:viscosity}, at \((x_0, \tau_0)\)
\[
(\partial_{\tau} - \mathcal{L}) \varphi_{\epsilon}|_{(x_0, \tau_0)} \geq 0.
\]

On the other hand, the assumptions on \(\varphi\) imply that
\[
\begin{split}
(\partial_{\tau} - \mathcal{L}) \varphi_{\epsilon} &= (1-\epsilon) (\partial_{\tau} - \mathcal{L}) \varphi + \epsilon^2 (\partial_{\tau} - \mathcal{L}) [e^{-\lambda\tau} x(2\pi-x)] \\
&\leq \epsilon^2  (\partial_{\tau} - \mathcal{L}) [e^{-\lambda\tau} x(2\pi-x)].
\end{split}
\]
From the argument above we know that \((x_0, \tau_0) \in [\bar{x}, 2\pi - \bar{x}] \times [0, \bar{\tau}]\) where \(x(x-2\pi) \geq \bar{x}(2\pi - \bar{x}) > 0\) has a positive lower bound. By the definition of \(\lambda\),
\[
\lambda x_0(2\pi - x_0) + \mathcal{L} [x(2\pi-x)]|_{x_0} > 0.
\]
Thus at \((x_0, \tau_0)\) we have
\[
(\partial_{\tau} - \mathcal{L}) \varphi_{\epsilon}|_{(x_0, t_0)} \leq -\epsilon^2 e^{-\lambda \tau_0} \left(\lambda x_0(2\pi - x_0) + \mathcal{L} [x(2\pi-x)]|_{x_0}\right) < 0
\]
which is our desired contradiction.

The last statement giving the curvature bound follows immediately from \Cref{prop:asymptotics} since both \(\varphi\) and \(\chordarcprofile\) have the same first order term.
\end{proof}

\section{Comparison Function}
\label{sec:comparison_function}

In this section, we seek suitable functions $\varphi$ satisfying the hypotheses of Theorem \ref{thm:comparison}. Our approach is to seek a one-parameter family $\varphi_c$ of such functions such that for any given curve there is a $c$ with $\varphi_c(x, 0) \leq \chordarcprofile(x, 0)$. Such functions are constructed as in \cite{MR2794630} by making a change of variables $x \mapsto \psi(x) = 2 \sin(x/2)$ which is the profile of a self-similar solution (i.e. round circle) to the flow while simultaneously changing the time parameter $\tau \mapsto e^{\tau}$. Then we seek a similarity solution of the resulting ODE. That is we make the ansatz,
\[
\varphi(x, \tau) = e^{\tau} \Phi(e^{-\tau} \psi(x)).
\]

\begin{thm}
\label{thm:comparison_function}
Let
\[
\varphi_c(x, \tau) = \frac{1}{c} e^{\tau} \arctan(c e^{-\tau} 2\sin(x/2)).
\]

Then there exists a $c \in \R$ such that
\[
\varphi_c(x, \tau(t)) \leq \chordarcprofile(x, t)
\]
for every $x, t$. Moreover,
\[
\sup_{p \in \S^1} \kappa(p, t) \leq 24\left(\frac{2\pi}{L(t)}\right)^2 Ce^{-a\tau(t)}.
\]
\end{thm}

\begin{proof}
We need to show two things:
\begin{enumerate}
\item Given \(\gamma_0\), there exists a \(c \in \R\) such that \(\varphi_c(x, 0) \leq \chordarcprofile(x, 0)\), and
\item the hypotheses of the Comparison \Cref{thm:comparison} are satisfied.
\end{enumerate}

Recall that the differential inequality in \Cref{thm:comparison} requires
\[
\partial_{\tau} \varphi - 4 \varphi'' + \left(\frac{L}{2\pi}\right)^2 (\varphi - x \varphi') \avg{\kappa^2} + \varphi' \frac{L}{2\pi} \int_{p_0}^{q_0} \kappa^2 ds \leq 0.
\]

The basic challenge is to express the (essentially unknown) curvature integrals in terms of the profile \(\chordarcprofile\). This can be done explicitly via \holder{}'s inequality:
\[
\int \kappa ds = \int \kappa \cdot 1 ds \leq \left(\int \kappa^2 ds\right)^{1/2} \left(\int 1 ds\right)^{1/2}
\]
so that using \(\int_{\gamma} \kappa ds = 2\pi\) and \(\int_{p_0}^{q_0} \kappa ds = \theta(p_0) - \theta(q_0)\) we have
\[
\left(\frac{L}{2\pi}\right)^2 \avg{\kappa^2} \geq 1
\]
and
\[
\frac{L}{2\pi} \int_{p_0}^{q_0} \kappa^2 ds \geq \frac{L}{2\pi \ell} (\theta(q_0) - \theta(p_0)^2 = \frac{4}{x} \arccos^2(\chordarcprofile').
\]
since
\[
\cos \theta(q_0) = - \cos \theta(p_0) = \chordarcprofile'.
\]

linearise around the self similar solution, \(\psi_0\). First observe that for the self-similar solution, namely the round circle \(\gamma_0\), we have \(\kappa_0 \equiv \tfrac{2\pi}{L_0}\). Then
\[
\avg{\kappa_0^2} = \frac{1}{L_0}\int_0^L \left(\frac{2\pi}{L_0}\right)^2 ds = \left(\frac{2\pi}{L_0}\right)^2
\]
is precisely the time scaling in \Cref{defn:timeprofile}. And
\[
\int_{p_0}^{q_0} \kappa_0^2 ds = \int_{p_0}^{q_0} \left(\frac{2\pi}{L_0}\right)^2 ds = \left(\frac{2\pi}{L_0}\right)^2 (q_0 - p_0) = \left(\frac{2\pi}{L_0}\right)^2 \ell(p_0, q_0).
\]
For a general

Our ansatz gives
\begin{align*}
\partial_{\tau} \varphi_c &= e^{\tau} \left(\Phi_c - e^{-\tau} \psi_0 \Phi_c'\right) \\
\varphi_c' &= \psi_0' \Phi_c' \\
\varphi_c'' &= e^{-\tau} (\psi_0')^2 \Phi_c'' + \psi_0'' \Phi_c'.
\end{align*}
Substituting the ansatz, the differential inequality becomes
\[
\begin{split}
0 &\geq e^{\tau} \left(\Phi_c - e^{-\tau} \psi_0 \Phi_c'\right) - 4\left(e^{-\tau} (\psi_0')^2 \Phi_c'' + \psi_0'' \Phi_c'\right) \\
&\quad + \left(\frac{L}{2\pi}\right)^2 (e^{\tau} \Phi - x \psi_0'\Phi') \avg{\kappa^2} + \psi_0'\Phi' \frac{L}{2\pi} \int_{p_0}^{q_0} \kappa^2 ds \\
&= - 4 e^{-\tau} (\psi_0')^2 \Phi_c'' + e^{\tau} \Phi + \left(\frac{L}{2\pi}\right)^2 e^{\tau} \Phi \\
&\quad + \left(-4 \psi_0'' + \psi_0' \frac{L}{2\pi}\int_{p_0}^{q_0} \kappa^2 ds - \left(\frac{L}{2\pi}\right)^2 \avg{\kappa^2} x\psi_0' - \psi_0\right) \Phi'
\end{split}
\]



Rearranging, this is equivalent to
\[
\begin{split}
0 &\leq 4 (e^{-\tau} \psi_0')^2 \Phi_c'' + \left(4\psi_0'' - \frac{L}{4\pi^2} \bar{Q}_0 x \psi_0' + \frac{1}{x} \psi_0'f_0(\Phi_c' \psi_0') \right) e^{-\tau} \Phi_c' + e^{-\tau} \psi_0 \Phi_c' + \frac{L}{4\pi^2} \bar{Q}_0 \Phi_c - \Phi_c \\
&= 4 (e^{-\tau} \psi_0')^2 \Phi_c'' + \left(\frac{L}{4\pi^2} \bar{Q}_0 - 1\right) \left(\Phi_c - e^{-\tau} \psi_0 \Phi_c'\right) + \frac{1}{x} \left(f_0(\Phi_c' \psi_0') - f_0(\psi_0')\right) \psi_0' e^{-\tau} \Phi_c'\\
&\quad + \left(4\psi_0'' + \frac{L}{4\pi^2} \bar{Q}_0 (\psi_0 - x \psi_0') + \frac{1}{x} \psi_0' f_0(\psi_0')\right) e^{-\tau} \Phi_c'.
\end{split}
\]
The self-similar solution $\psi_0$ satisfies
\[
0 = \partial_t \psi_0 \leq 4\psi_0'' + \frac{L}{4\pi^2} \bar{Q}_0 (\psi_0 - x \psi_0') + \frac{1}{x} \psi_0' f_0(\psi_0').
\]
Thus it is sufficient to prove
\[
0 \leq 4 (e^{-\tau} \psi_0')^2 \Phi_c'' + \left(\frac{L}{4\pi^2} \bar{Q}_0 - 1\right) \left(\Phi_c - e^{-\tau} \psi_0 \Phi_c'\right) + \frac{1}{x} \left(f_0(\Phi_c' \psi_0') - f_0(\psi_0')\right) \psi_0' e^{-\tau} \Phi_c'.
\]

We make one final estimation linearising $f_0$ around the self similar solution: by taking a Taylor expansion around $\psi_0'$ with second order remainder, the concavity of $f_0$ implies the remainder is non-negative and hence
\[
f_0(\Phi_c' \psi_0') - f_0(\psi_0') \geq f_0'(\psi_0')(\Phi_c'\psi_0' - \psi_0') = \psi_0' f_0'(\psi_0')(\Phi_c' - 1).
\]
Let us also write $u = e^{-\tau} \psi_0$ for the argument to $\Phi_c$ so that we finally reduce to
\begin{equation}
\label{eq:Phi_ode}
\begin{split}
0 &\leq 4 (e^{-\tau} \psi_0')^2 \Phi_c'' + \frac{1}{x} \psi_0' f_0'(\psi_0')(\Phi_c' - 1) \psi_0' e^{-\tau} \Phi_c' \\
&\quad + \left(\frac{L}{4\pi^2} \bar{Q}_0 - 1\right) \left(\Phi_c - u \Phi_c'\right) \\
&= 4 e^{-\tau} \frac{(\psi_0')^2}{\psi_0} \left(u \Phi_c'' + \frac{1}{4x} \psi_0 f_0'(\psi_0')(\Phi_c' - 1) \Phi_c'\right) \\
&\quad + \left(\frac{L}{4\pi^2} \bar{Q}_0 - 1\right) \left(\Phi_c - u \Phi_c'\right).
\end{split}
\end{equation}

\textbf{What is needed next (see remark \ref{rem:smooth_case} below for how it works in the smooth case):}
The concavity of $\Phi_c$ implies $\left(\Phi_c - u \Phi_c'\right) \geq 0$ and so if $\frac{L}{4\pi^2} \bar{Q}_0 \geq 1$, if suffices to show
\[
u \Phi_c'' + \frac{1}{4x} \psi_0 f_0'(\psi_0')(\Phi_c' - 1) \Phi_c' \geq 0.
\]
For the asymptotics, we require
\[
\Phi(u) = u + \bigo(u^2).
\]
Then concavity of $\Phi$, positivity of $\Phi$ and $\Phi(0) = 0$ also implies,
\[
0 \leq \Phi'(u) \leq 1.
\]
Thus we need
\[
\frac{1}{4x} \psi_0 f_0'(\psi_0') \leq -\alpha
\]
for some $\alpha > 0$ which would reduce finally to the ODE,
\begin{equation}
\label{eq:comparison_ode}
u \Phi_c'' - \alpha (\Phi_c' - 1) \Phi_c' \geq 0.
\end{equation}

This is a simple first order ODE for $\Phi_c'$ with solution,
\[
\Phi_c'(u) = \frac{1}{1 \pm (cu)^{\alpha}}
\]
with $c > 0$. Coupled with the intial condition $\Phi_c(0) = 0$ we then have
\begin{equation}
\label{eq:Phi}
\Phi_c (u) = c \int_0^{u/c} \frac{dv}{1 \pm v^{\alpha}}.
\end{equation}
By defining
\[
\Phi_0(u) = \int_0^{u} \frac{dv}{1 \pm v^{\alpha}}
\]
we have
\[
\Phi_c(u) = c \Phi_0(u/c)
\]
and
\[
\varphi_c(x, \tau) = c e^{\tau} \Phi_0\left(\tfrac{u}{ce^{\tau}}\right).
\]

We also require that $\Phi_C$ satisfies the asymptotics
\begin{equation}
\label{eq:comparison_asymptotics}
\Phi_c(u) = u + C_2 u^2 + \bigo(u^3)
\end{equation}
for $C_2 \ne 0$ (very important!). Then since the self-simlar solution already has the correct asymptotics,
\[
\psi_0(x) = \frac{\sqrt{3}}{2} x + a_2 x^2 + \bigo(x^3)
\]
we obtain
\[
\begin{split}
\phi_c(x, t) &= e^t \Phi_C (e^{-t} \psi_0(x)) = \frac{\sqrt{3}}{2} x + a_2 x^2 + \frac{3}{4} C_2 e^{-t} x^2 + \bigo(x^3) \\
& =  \frac{\sqrt{3}}{2} x + (a_2  + \frac{3}{4} C_2 e^{-t}) x^2 + \bigo(x^3)
\end{split}
\]
automatically has the correct first order asymptotics and the second order asymptotics converges exponentially fast to $a_2$, the second order term of $\psi_0$.

For $\Phi_c$ given in equation \eqref{eq:Phi} we have
\begin{align*}
\Phi_c(0) &= 0 \\
\Phi_c'(0) &= 1 \\
\Phi_c''(0) &= \mp \left.\frac{\alpha c^{-\alpha} u^{\alpha-1}}{(1 \pm (u/c)^{\alpha})^2}\right|_{u=0} = \begin{cases}
0, & \alpha > 1 \\
\mp \tfrac{1}{c}, & \alpha = 1 \\
\mp \infty, &  0 < \alpha < 1.
\end{cases}
\end{align*}

\textbf{So this will not work unless $\alpha=1$!}

Since we require $\Phi_c$ to be concave and $\alpha=1$ we have
\[
\Phi_c(u) =  \frac{1}{c} \int_0^{cu} \frac{dv}{1 + v} = \frac{1}{c} \ln(1 + cu).
\]
with any $c>0$. Then our proposed comparison has exactly the right asymptotics:
\[
\varphi_c(x, t) = \frac{\sqrt{3}}{2} x + (a_2 - \frac{3c}{8} e^{-t}) x^2 + \bigo(x^3).
\]

The function $\Phi_c$ also has the important properties
\[
\lim_{c\to 0} \Phi_c (u) = 0, \quad \lim_{c\to \infty} \Phi_c (u) = u.
\]
The former limit implies
\[
\lim_{c\to 0} \varphi_c (x, t) = 0
\]
allowing us to make the barrier $\varphi_c$ arbitrarily small for the initial comparison, while the latter limit implies
\[
\lim_{t\to \infty} \varphi_c(x, t) = \lim_{t\to\infty} c e^t \Phi_0(\psi_0(x)/ce^t) = \psi_0(x)
\]
so that our barrier limits to the self-similar profile. In this way the barrier interpolates between the fully collapsed situation $\chordarcprofile \equiv 0$, and the self-similar profile forcing anything in between to be attracted to the self-similar profile.
\end{proof}

\begin{rem}
 Note that in the theorem, if we begin with the self-similar solution, we could simply take $\Phi_c = \id$ whence $\Phi_c'' = \Phi_c - u \Phi' = \Phi_c' - 1 = 0$. The the right hand side of equation \eqref{eq:Phi_ode} vanishes and so the inequality is sharp with equality attained on the self-similar solution. The philosophy guiding the estimates is that all estimates should be equalities on the self-simliar solution.
\end{rem}

\begin{rem}
\label{rem:smooth_case}
In the smooth case \cite{MR2794630}, we have
\[
\psi_0 = 2 \sin(x/2).
\]
and \holder's inequality gives,
\[
\frac{L}{4\pi^2} \bar{Q} \geq 1, \quad \frac{L}{2\pi} \Delta Q \geq \frac{4}{x} \arccos^2(\psi_0'\Phi_c').
\]
The self-similar solution is the circle, hence the curvature is constant and we have equality in \holder{}'s inequality yielding,
\[
\frac{L}{4\pi^2} \bar{Q}_0 = 1, \quad \frac{L}{2\pi} \Delta Q_0 = \frac{2\pi}{L} \ell = x = \frac{4}{x} \arccos^2(\psi_0')
\]
since $\psi_0'(x) = \cos(x/2)$. Thus $f_0 = 4 \arccos^2$ and
\[
\frac{1}{4x} \psi_0 f_0'(\psi_0') = - \frac{1}{4x} 2 \sin(x/2) \frac{8\arccos(\cos(x/2))}{\sqrt{1 - \cos^2(x/2)}} = - \frac{1}{4x} 2 \sin(x/2) \frac{8x}{2\sin(x/2)} = -2
\]

In this setting, equation \eqref{eq:self_similar_comparison} becomes
\[
\partial_t \varphi_c \leq 4 \varphi_c'' + \varphi_c - x\varphi_c' + \frac{1}{x} \varphi_c' 4\arccos^2(\varphi_c').
\]
while equation \eqref{eq:Phi_ode} becomes
\[
\begin{split}
0 &\leq 4 e^{-\tau} \frac{(\psi_0')^2}{\psi_0} \left(u \Phi_c'' + \frac{1}{4x} \psi_0 f_0'(\psi_0')(\Phi_c' - 1) \Phi_c'\right) \\
&= 4 e^{-\tau} \frac{(\psi_0')^2}{\psi_0} \left(u \Phi_c'' - 2 (\Phi_c' - 1) \Phi_c'\right)
\end{split}
\]
and we obtain an ODE inequality for $\Phi$.

The comparison function chosen in \cite{MR2794630},
\[
\Phi_c (u) = 2 c \arctan(u/2c)
\]
solves
\[
\begin{cases}
u \Phi_c'' - 2 (\Phi_c' - 1) \Phi_c' &= 0 \\
\Phi_c(0) &= 0
\end{cases}
\]
for any choice of $c \ne 0$ and so equation \eqref{eq:Phi_ode} becomes an equality for this particular choice.

Moroever we have the asymptotics
\[
\Phi_c(u) = u + C_3 u^3 + \bigo(u^4)
\]
which automatically gives $\phi_c(x, t) = e^t \Phi_c (e^{-t} \psi_0(x))$ the correct asymptotics noting that in the smooth case, the profile has the asymptotics
\[
\chordarcprofile(x) = x + C_3 x^3 + \bigo(x^4)
\]
with the third order term exponentially decaying to the third order term of the self similar profile.
\end{rem}



\printbibliography

\end{document}
